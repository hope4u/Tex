\section{the Receiver}
In this project we were mainly looking into optimizing on the transmitter side of the channel but the receiver structure has a huge impact at the performance of the system too.
In this section we will use two different receiver structures:
\begin{enumerate}
	\item Linear MMSE Receiver\\
	//formula
	\item V-BLAST MMSE or SIC with MMSE (MMSE-SIC)\\
	//formula (algorithm)\\
	Generally speaking the SIC receivers provide better Throughput but are more complicated.
\end{enumerate}

\subsection{Linear MMSE}
The goal of the MMSE equalizer is to minimize the mean error at the filter output. We use $\mathbf{K}_y = \mathbf{HH}^H + \mathbf{K}_n$ and equal power allocation at the transmitter. Then we can write the error covariance matrix as
\begin{align}
	\mathbf{K}_e &= \mathbf{GK}_y\mathbf{G}^H - \mathbf{GH} - \mathbf{H}^H\mathbf{G}^H + \mathbf{I}_N\\
	&=\Biggr(\mathbf{GK}_y - \mathbf{H}^H\Biggl) \mathbf{K}_y^{-1} \Biggr(\mathbf{GK}_y - \mathbf{H}^H\Biggl)^H - \mathbf{H}^H\mathbf{K}_y^{-1}\mathbf{H}+\mathbf{I}_N.
\end{align}
and since $\mathbf{K}_y$ is a non-negative definite matrix we need to minimize $\mathbf{GK}_y-\mathbf{H}^H\bigl=\mathbf{0}$ with $\mathbf{K}_y = \sigma_n^2\mathbf{I}_N$. This leads us to the filter matrix $\mathbf{G}$ as
\begin{align}
	\mathbf{G} &= \mathbf{H}^H\mathbf{K}_y^{-1}\\
	&= \mathbf{H}^H\Biggr(\mathbf{HH}^H + \sigma_n^2\mathbf{I}_N\Biggl)^{-1}\\
	&=\Biggr(\mathbf{HH}^H + \sigma_n^2\mathbf{I}_N\Biggl)^{-1}\mathbf{H}^H.
\end{align}
where the last equation uses the matrix inversion lemma.\\
Now we use $\tilde{x} = \mathbf{Gy}$ to estimate the sent vector $\mathbf{x}$. This gives us
\begin{equation}
	\mathbf{K}_{e,MMSE} = \mathbf{I}_N - \mathbf{H}^H\mathbf{K}_y^{-1}\mathbf{H} = \sigma_n^2\Biggr(\mathbf{H}^H\mathbf{H} + \sigma_n^2\mathbf{I}_N\Biggl)^{-1}
\end{equation}
for the error covariance matrix and a channel capacity of
\begin{equation}
	C = -\sum_{i=1}^{N}{\log_2{\Biggr(\Bigr[\mathbf{K}_{e,MMSE}\Bigl]_{i,i}^{-1}\Biggl)}}.
\end{equation}
The corresponding per-user SINR is then calculated with
\begin{equation}
	\text{SINR}_{i} = \frac{1}{\Bigr[\mathbf{K}_{e,MMSE}\Bigl]_{i,i}^{-1}} - 1.
\end{equation}

\subsection{MMSE-SIC}
We can use the result of one decorrelator filter to aid the others which results in a successive cancellation strategy: we decode $x_1[m]$ first and then subtract the decoded stream from the received vector. So the second decoder sees only the interference of the streams $x_3,\dots,x_m$.But be aware of error propagation, this means that an error in decoder 2 leads to erroneous results in all subsequent decoders.\\
The V-BLAST gives us
\begin{equation}
	R<\log_2{\det{\Biggr(\mathbf{I}_N+\frac{1}{N_0}\mathbf{H}\mathbf{K_x}\mathbf{H}^H\Biggl)}} bits/s/Hz
\end{equation}
with $\mathbf{K_x} = \diag{P_1,\dots,P_M}$ as the covariance matrix of the transmitted signal. Which leads to a capacity of
\begin{equation}
	C = \max_{\Tr{|\mathbf{K_x}|}\leq P}{\log_2{\det{\Biggr(\mathbf{I}_N+\frac{1}{N_0}\mathbf{H}\mathbf{K_x}\mathbf{H}^H\Biggl)}}} bits/s/Hz
\end{equation}
which is the capacity of the fas fading MIMO channel.\\
in the case of CSIT we already discussed the optimal solution as the waterfilling. But with no CSI on the transmitter side //tse proves that an equal Power allocation, in other words equal power on each transmit antenna, gives us the best performance
\begin{equation}
	\mathbf{K_x} = \Biggr(\frac{P}{M}\Biggl)\mathbf{I}_N
\end{equation}
which gives us a resulting capacity of
\begin{equation}
	C =\log_2{\det{\Biggr(\mathbf{I}_N+\frac{
	\text{SNR}}{M}\mathbf{HH}^H\Biggl)}}.
\end{equation}
We see a power loss in comparison to full CSI of $M$ which is achieved by using transmit beaming.\\
In general the decoding order of a this SIC receivers has no impact on the performance, but in our work we always decode the channel which provides the highest MMSE-SINR first. With this decoding order we can guarantee a solution for the later discussed max-min algorithm.
// picture\\
// code