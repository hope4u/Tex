\section{optimizing for SINR target}
\subsection{fodor}

\subsection{The Algorithm}
\begin{enumerate}
	\item calculate the effective interference for MMSE processing:
		\begin{equation}
			\begin{aligned}
				\zeta_{k,s} = \Biggl\{\Biggl(d^{-\rho}_{k,k}\chi_{k,k}H^H_{k,k}\biggl(\sum_{j\neq k}{P_jd^{-\rho}_{k,j}\chi_{k,j}H_{k,j}T_jT_j^HH^H_{k,j}}\\
				+N_t\sigma^2_nI\biggr)^{-1}H_{k,k}+\frac{1}{P_k}I\Biggr)^{-1}\Biggr\}^{(s,s)}
			\end{aligned}	
		\end{equation}

	\item calculate the optimal loading matrix
		\begin{equation}
			(T_k)^{(s,s)} = \sqrt{\frac{\zeta_{k,s}N_t}{\sum_{j=1}^{N_t}\zeta_{k,j}}}\;\forall s\in[1,N_t]
		\end{equation}

	\item calculate used Power
		\begin{equation}
			P_k = \frac{\zeta_{k,s}}{\vert(T_k)^{(s,s)}\vert^2}(\gamma_{tgt}+1)\;\forall k
		\end{equation}

	\item[n.] until no more change

\end{enumerate}

\subsection{analytical gradient}
We calculate the power minimization on the basis of the analytical gradient for throughput maximization.
\begin{equation}
	\mathbf{P} = \arg{\min_{\begin{matrix}
		P_k\geq 0\,\forall P_k\in \diag{\mathbf{P}}\\
		R_k = R_{tgt}\,\forall k
		\end{matrix}}{\sum_k{P_k}}}
\end{equation}The idea here was to minimize the difference between the target Rates and the calculated rate for the channel, only by changing the diagonal Values of $\mathbf{P}$.
So our problem gets reformed into
\begin{equation}
	\mathbf{P} = \arg{\max_{P_k\geq 0\,\forall P_k\in \diag{\mathbf{P}}}{\sum_k{\bigr(R_k-R_{tgt}\bigl)}}}
\end{equation}
Again we use $\mathbf{X}^2 = \mathbf{P}$ to make sure to only get positive power Values but since we do not care about some power power budget $P_{tot}$ we can removed the condition $\trace{\mathbf{P}}\leq P_{tot}$. $\mathbf{\Phi}$ then becomes
\begin{equation}
	\mathbf{\Phi} = \mathbf{X}\mathbf{H}^H\frac{1}{\sigma^2}\mathbf{H} \mathbf{X} + \mathbf{I_M}
\end{equation}
and the for the Rate of user $k$
\begin{equation}
	R_k = -\log_2\Biggr(\mathbf{\Phi}_{k,k}^{-1}\Biggl) = -\log_2\Biggr(\biggr[\mathbf{X}\mathbf{H}^H\frac{1}{\sigma^2}\mathbf{H} \mathbf{X} + \mathbf{I_M} \biggl]_{k,k}^{-1}\Biggl)
\end{equation}
With $p=2$ we calculate the distance such that the sign does not mess up the gradient and the norm to be able to reuse the code from the throughput maximization algorithm, we get:
\begin{equation}
	\nabla \bigr\Vert{(R(\mathbf{P}))\bigl\Vert}_2 = \nabla \Biggr[ \Biggr(\sum_k{(R_k-R_{tgt})^2}\Biggl)^{1/2} \Biggl]
\end{equation}
and for the partial derivative
\begin{align}
	\partial_{x_j} \bigr\Vert{(R(\mathbf{P}))\bigl\Vert}_2
	&=\partial_{x_j}\Biggr[\Biggr(\sum_k{(R_k-R_{tgt})^2}\Biggl)^{1/2}\Biggl]\\
	&=\frac{1}{2}\Biggr(\sum_k{(R_k-R_{tgt})^2}\Biggl)^{-1/2} \cdot \sum_k{\Biggr[2(R_k-R_{tgt}) \cdot \partial_{x_j}(R_k-R_{tgt})\Biggl]}.
\end{align}
where
\begin{align}
	\partial_{x_j}(R_k-R_{tgt}) &= \partial_{x_j}(R_k)\\
	&= \partial_{x_j}\bigr(-\log_2{\mathbf{\Phi}_{k.k}^{-1}}\bigl)\\
	&=\frac{-1}{\log{2}}\frac{1}{\mathbf{\Phi}_{k,k}^{-1}} \cdot \partial_{x_j}\bigr(\mathbf{\Phi}_{k,k}^{-1}\bigl)
\end{align}
\begin{equation}
	\partial_{x_j}\bigr(\mathbf{\Phi}_{k,k}^{-1}\bigl) = \Biggr(-\mathbf{\Phi}^{-1} \cdot \partial_{x_j}\bigr(\mathbf{\Phi}^{-1}\bigl) \cdot \mathbf{\Phi}^{-1}\Biggl)_{k,k}.
\end{equation}
and the derivative of $\mathbf{\Phi}$ as
\begin{align}
	\partial_{x_j}\bigr(\mathbf{\Phi}^{-1}\bigl) &= \partial_{x_j} \biggr(\mathbf{X}\mathbf{H}^H\frac{1}{\sigma^2}\mathbf{H} \mathbf{X} + \mathbf{I_M} \biggl)\\
	&= \partial_{x_j} \biggr(\mathbf{X}\mathbf{H}^H\frac{1}{\sigma^2}\mathbf{H} \mathbf{X}\biggl)\\
	&= \mathbf{E}_{j,j}\mathbf{H}^H\frac{1}{\sigma^2}\mathbf{HX} + \mathbf{XH}^H\frac{1}{\sigma^2}\mathbf{HE}_{j,j}
\end{align}
and finally
\begin{equation}
	\begin{aligned}
		\partial_{X_j}(R_k) = \frac{1}{2}\Biggr(\sum_k{(R_k-R_{tgt})^2}\Biggl)^{-1/2}\cdot\sum_k{\Biggr[2(R_k-R_{tgt})\cdot\frac{1}{\log{2}\cdot\mathbf{\Phi}_{k,k}^{-1}}\cdot\\\cdot\Biggr(-\mathbf{\Phi}^{-1} \cdot\Biggr( \mathbf{E}_{j,j}\mathbf{H}^H\frac{1}{\sigma^2}\mathbf{HX} + \mathbf{XH}^H\frac{1}{\sigma^2}\mathbf{HE}_{j,j}\Biggl) \cdot \mathbf{\Phi}^{-1}\Biggl)_{k,k}\Biggl]}.
	\end{aligned}
\end{equation}